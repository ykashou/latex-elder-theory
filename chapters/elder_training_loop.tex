\section{Elder Training Loop}

\subsection{Complete Algorithm for Elder Training}

The Elder training loop represents the highest level of learning in our hierarchical system, where universal principles are extracted from cross-domain knowledge. Below, we present the complete mathematical formulation of the Elder training algorithm.

\begin{algorithm}
\caption{Elder Training Loop}
\begin{algorithmic}[1]
\State \textbf{Input:} Set of domains $\mathcal{D} = \{D_1, D_2, \ldots, D_M\}$
\State \textbf{Input:} Dataset for each domain $\mathcal{X}_i, \mathcal{Y}_i$ for $D_i \in \mathcal{D}$
\State \textbf{Input:} Initial Elder parameters $\theta_{\text{Elder}}^{(0)} \in \elderparams$
\State \textbf{Input:} Initial Mentor parameters $\{\theta_{\text{M},i}^{(0)}\}_{i=1}^M \subset \mentorparams$
\State \textbf{Input:} Initial Erudite parameters $\{\theta_{\text{E},i,j}^{(0)}\}_{i=1,j=1}^{M,N_i} \subset \eruditeparams$
\State \textbf{Input:} Learning rates $\eta_{\text{Elder}}, \eta_{\text{M}}, \eta_{\text{E}}$
\State \textbf{Input:} Number of epochs $T$
\State \textbf{Input:} Batch size $B$

\For{$t = 1$ to $T$}
    \State $\nabla_{\theta_{\text{Elder}}} \mathcal{L}_{\text{Elder}} \gets \mathbf{0}$ \Comment{Initialize Elder gradient}
    
    \For{each domain $D_i \in \mathcal{D}$}
        \State $\nabla_{\theta_{\text{M},i}} \mathcal{L}_{\text{M}} \gets \mathbf{0}$ \Comment{Initialize Mentor gradient for domain $D_i$}
        
        \For{$j = 1$ to $N_i$} \Comment{For each task in domain $D_i$}
            \State $\nabla_{\theta_{\text{E},i,j}} \mathcal{L}_{\text{E}} \gets \mathbf{0}$ \Comment{Initialize Erudite gradient for task $j$}
            
            \State Sample batch $\{(x_k, y_k)\}_{k=1}^B$ from $(\mathcal{X}_{i,j}, \mathcal{Y}_{i,j})$
            
            \For{$k = 1$ to $B$}
                \State $z_{i,j,k} \gets f_{\theta_{\text{E},i,j}}(x_k)$ \Comment{Erudite forward pass}
                \State $\mathcal{L}_{\text{E},k} \gets \eruditeloss(z_{i,j,k}, y_k)$ \Comment{Compute Erudite loss}
                \State $\nabla_{\theta_{\text{E},i,j}} \mathcal{L}_{\text{E}} \mathrel{+}= \frac{1}{B} \nabla_{\theta_{\text{E},i,j}} \mathcal{L}_{\text{E},k}$ \Comment{Accumulate Erudite gradient}
            \EndFor
            
            \State $p_{\text{M},i,j} \gets \mentorreflection_{\theta_{\text{M},i}}(\theta_{\text{E},i,j})$ \Comment{Mentor reflection on Erudite}
            \State $\mathcal{L}_{\text{M},i,j} \gets \mentorloss(p_{\text{M},i,j}, \{\theta_{\text{E},i,l}\}_{l=1}^{N_i})$ \Comment{Compute Mentor loss}
            \State $\nabla_{\theta_{\text{M},i}} \mathcal{L}_{\text{M}} \mathrel{+}= \frac{1}{N_i} \nabla_{\theta_{\text{M},i}} \mathcal{L}_{\text{M},i,j}$ \Comment{Accumulate Mentor gradient}
        \EndFor
        
        \State $p_{\text{Elder},i} \gets \elderreflection_{\theta_{\text{Elder}}}(\theta_{\text{M},i})$ \Comment{Elder reflection on Mentor}
        \State $\mathcal{L}_{\text{Elder},i} \gets \elderloss(p_{\text{Elder},i}, \{\theta_{\text{M},l}\}_{l=1}^{M})$ \Comment{Compute Elder loss}
        \State $\nabla_{\theta_{\text{Elder}}} \mathcal{L}_{\text{Elder}} \mathrel{+}= \frac{1}{M} \nabla_{\theta_{\text{Elder}}} \mathcal{L}_{\text{Elder},i}$ \Comment{Accumulate Elder gradient}
    \EndFor
    
    \State $\theta_{\text{Elder}}^{(t)} \gets \theta_{\text{Elder}}^{(t-1)} - \eta_{\text{Elder}} \nabla_{\theta_{\text{Elder}}} \mathcal{L}_{\text{Elder}}$ \Comment{Update Elder parameters}
    
    \For{each domain $D_i \in \mathcal{D}$}
        \State $\theta_{\text{M},i}^{(t)} \gets \theta_{\text{M},i}^{(t-1)} - \eta_{\text{M}} \nabla_{\theta_{\text{M},i}} \mathcal{L}_{\text{M}}$ \Comment{Update Mentor parameters}
        
        \For{$j = 1$ to $N_i$}
            \State $\theta_{\text{E},i,j}^{(t)} \gets \theta_{\text{E},i,j}^{(t-1)} - \eta_{\text{E}} \nabla_{\theta_{\text{E},i,j}} \mathcal{L}_{\text{E}}$ \Comment{Update Erudite parameters}
        \EndFor
    \EndFor
\EndFor

\State \textbf{Return:} $\theta_{\text{Elder}}^{(T)}, \{\theta_{\text{M},i}^{(T)}\}_{i=1}^M, \{\theta_{\text{E},i,j}^{(T)}\}_{i=1,j=1}^{M,N_i}$
\end{algorithmic}
\end{algorithm}

\subsection{Elder Manifold Update Phase}

A critical aspect of the Elder training loop is the manifold update phase, which occurs after gradient computation but before parameter updates. This phase ensures that the knowledge state maintains its holomorphic structure on the Elder Manifold $\mathcal{E}_{\mathcal{M}}$.

\begin{algorithm}
\caption{Elder Manifold Update}
\begin{algorithmic}[1]
\State \textbf{Input:} Current Elder knowledge point $p \in \mathcal{E}_{\mathcal{M}}$
\State \textbf{Input:} Elder gradient $\nabla_{\theta_{\text{Elder}}} \mathcal{L}_{\text{Elder}}$
\State \textbf{Input:} Learning rate $\eta_{\text{Elder}}$

\State $p^* \gets \mathcal{M}(p)$ \Comment{Apply Holomorphic Mirror function}
\State $v \gets \text{parallel\_transport}(\mathcal{J}(p^*) - p)$ \Comment{Compute displacement vector}
\State $p_{\text{new}} \gets \exp_p(\eta_{\text{Elder}} \cdot v)$ \Comment{Update via exponential map}

\State \textbf{Return:} $p_{\text{new}}$
\end{algorithmic}
\end{algorithm}

\subsection{Knowledge Transformation via Holomorphic Flow}

The final component of the Elder training loop involves knowledge transformations through holomorphic flows on the manifold, ensuring that universal principles evolve coherently.

\begin{algorithm}
\caption{Holomorphic Knowledge Flow}
\begin{algorithmic}[1]
\State \textbf{Input:} Current Elder knowledge state $p \in \mathcal{E}_{\mathcal{M}}$
\State \textbf{Input:} Holomorphic vector field $X: \mathcal{E}_{\mathcal{M}} \rightarrow T\mathcal{E}_{\mathcal{M}}$
\State \textbf{Input:} Time step $\Delta t$

\State $\frac{dp}{dt} = X(p)$ \Comment{Differential equation for knowledge flow}
\State $p_{\Delta t} \gets p + \int_0^{\Delta t} X(p(s)) ds$ \Comment{Integrate flow equation}

\State \textbf{Return:} $p_{\Delta t}$
\end{algorithmic}
\end{algorithm}

\subsection{Cross-Domain Knowledge Integration}

The Elder's primary function is to integrate knowledge across domains, expressed mathematically through the following operations:

\begin{equation}
\begin{aligned}
\mathcal{K}_{\text{Elder}} &= \int_{\mathcal{D}} \kappa(D_i, D_j) \cdot \mathcal{T}(\theta_{\text{M},i}, \theta_{\text{M},j}) d\mu(D_i) d\mu(D_j) \\
\end{aligned}
\end{equation}

Where $\kappa$ is the domain similarity kernel, $\mathcal{T}$ is the knowledge transfer operator, and $\mu$ is a measure on the domain space $\mathcal{D}$.

In practice, this integration is computed as:

\begin{equation}
\mathcal{K}_{\text{Elder}} = \sum_{i=1}^M \sum_{j=1}^M w_{i,j} \cdot \mathcal{T}(\theta_{\text{M},i}, \theta_{\text{M},j})
\end{equation}

Where $w_{i,j} = \kappa(D_i, D_j) / \sum_{k,l} \kappa(D_k, D_l)$ are the normalized weights.

This knowledge integration forms the core of the Elder's ability to extract universal principles that apply across diverse domains, enabling the system to achieve true cross-domain transfer learning.