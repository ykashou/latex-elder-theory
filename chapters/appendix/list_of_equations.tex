\chapter{List of Equations}

This appendix provides a comprehensive reference to all numbered equations throughout the Elder Theory manuscript, organized by chapter for easy reference.

\section{Overview}

The equations in this document form the mathematical foundation of Elder Theory, encompassing:
\begin{itemize}
    \item Heliomorphic function definitions and properties
    \item Orbital mechanics formulations
    \item Information-theoretic measures
    \item Thermodynamic relationships
    \item Learning algorithms and convergence criteria
\end{itemize}

Each equation is listed with its number, mathematical content, brief description, and page reference for quick navigation.

\section{Key Equations by Category}

\subsection{Heliomorphic Function Definitions}

\begin{description}
\item[Equation 2.1] Heliomorphic Function Definition: $f: \mathbb{C} \to \mathbb{C}$ with radial-phase coupling
\item[Equation 2.5] Composition Rule: $(f \circ g)(z) = f(g(z))$ preserving heliomorphic properties
\item[Equation 3.2] Differential Heritage: $\frac{\partial f}{\partial r} = \frac{1}{r}\frac{\partial f}{\partial \phi}$
\end{description}

\subsection{Orbital Mechanics}

\begin{description}
\item[Equation 16.3] Elder Gravitational Field: $\mathcal{G}_E(\mathbf{r}) = \frac{G m_E}{|\mathbf{r} - \mathbf{r}_E|^2}$
\item[Equation 17.1] Orbital Parameter Trajectories: $\frac{d^2\mathbf{r}_i}{dt^2} = \mathcal{G}_{\text{total}}(\mathbf{r}_i)$
\item[Equation 18.4] Phase Velocity: $\omega_i = \omega_0 + \alpha \sin(\phi_i - \phi_{\text{ref}})$
\end{description}

\subsection{Thermodynamic Formulations}

\begin{description}
\item[Equation 20.2] Elder Phase Space Entropy: $S = -k \sum_{\mu} p(\mu) \ln p(\mu)$
\item[Equation 20.7] Fokker-Planck Equation: $\frac{\partial p(\mu, t)}{\partial t} = -\nabla \cdot (p(\mu, t) \vec{F}(\mu)) + D \nabla^2 p(\mu, t)$
\item[Equation 20.12] Reverse Diffusion Learning: $\frac{\partial p(\mu, t)}{\partial t} = -D \nabla^2 p(\mu, t) + \nabla \cdot (p(\mu, t) \nabla \ln q(\mu))$
\end{description}

\subsection{Information Theory}

\begin{description}
\item[Equation 58.1] Mutual Information Transfer: $I(X;Y) = \sum_{x,y} p(x,y) \log \frac{p(x,y)}{p(x)p(y)}$
\item[Equation 54.3] Information Capacity: $C = \max_{p(x)} I(X;Y)$
\item[Equation 55.2] Entropy Dynamics: $\frac{dH}{dt} = -\frac{\partial \mathcal{L}}{\partial H}$
\end{description}

\subsection{Learning and Convergence}

\begin{description}
\item[Equation 31.5] Resonance-Amplified Update: $\theta_{t+1} = \theta_t - \eta \nabla \mathcal{L} \cdot R(\phi_t)$
\item[Equation 60.2] Convergence Guarantee: $\|\theta_t - \theta^*\| \leq \epsilon e^{-\lambda t}$
\item[Equation 52.1] PAC Learning Bound: $P(|R(\hat{h}) - R^*(h)| \leq \epsilon) \geq 1 - \delta$
\end{description}

\section{Complete Equation Index}

The following list includes all numbered equations from the document with automatic page references:

\listofequations

\section{Key Equation Categories}

\subsection{Fundamental Heliomorphic Relationships}
The core mathematical structures that define heliomorphic functions and their properties, including composition rules, differential operators, and completeness theorems.

\subsection{Orbital Dynamics}
Equations governing the movement and interaction of entities within the Elder Heliosystem, including gravitational forces, phase relationships, and stability criteria.

\subsection{Thermodynamic Formulations}
Mathematical descriptions of entropy, temperature, and energy flow within the system, particularly relating to the reverse diffusion learning process.

\subsection{Information Theory}
Measures of information content, mutual information, and knowledge transfer between hierarchical levels in the Elder architecture.

\subsection{Learning and Convergence}
Algorithmic formulations for training procedures, convergence guarantees, and performance bounds.

\section{Cross-References}

For detailed derivations and applications of these equations, please refer to the respective chapters and sections where they appear. The page numbers provided in the equation list will direct you to the original context and explanation.