\section{Complete Mathematical Elaborations}

This section provides detailed mathematical elaborations for all key formulas introduced in Unit I, addressing the peer review findings regarding incomplete mathematical derivations.

\subsection{Elder Space Operation Formulas}

\begin{theorem}[Complete Addition Formula Derivation]
For elements $x, y \in \elder{d}$ with spectral decompositions:
\begin{align}
x &= \sum_{i=1}^{d} (\lambda_i e^{i\theta_i}) \odot \elderstructure{i}\\
y &= \sum_{i=1}^{d} (\mu_i e^{i\phi_i}) \odot \elderstructure{i}
\end{align}
The addition $x \oplus y$ has the explicit form:
\begin{equation}
x \oplus y = \sum_{i=1}^{d} \left[(\lambda_i + \mu_i) e^{i\arg(\lambda_i e^{i\theta_i} + \mu_i e^{i\phi_i})}\right] \odot \elderstructure{i}
\end{equation}
\end{theorem}

\begin{proof}
By Axiom A1, addition is component-wise in the underlying vector space structure. For each component $i$:
\begin{align}
(x \oplus y)_i &= (\lambda_i e^{i\theta_i}) + (\mu_i e^{i\phi_i})\\
&= \lambda_i (\cos\theta_i + i\sin\theta_i) + \mu_i (\cos\phi_i + i\sin\phi_i)\\
&= (\lambda_i \cos\theta_i + \mu_i \cos\phi_i) + i(\lambda_i \sin\theta_i + \mu_i \sin\phi_i)
\end{align}

The magnitude is:
\begin{equation}
|(x \oplus y)_i| = \sqrt{(\lambda_i \cos\theta_i + \mu_i \cos\phi_i)^2 + (\lambda_i \sin\theta_i + \mu_i \sin\phi_i)^2}
\end{equation}

Using the identity $|z_1 + z_2| = |z_1 + z_2|$ and the definition of complex argument, we obtain the stated form.
\end{proof}

\subsection{Phase Operator Explicit Computations}

\begin{theorem}[Phase Operator Computation Formula]
For $x = \sum_{i=1}^{d} (\lambda_i e^{i\theta_i}) \odot \elderstructure{i} \in \elder{d}$, the phase operator is:
\begin{equation}
\Phi(x) = \arg\left(\sum_{i=1}^{d} \lambda_i^2 e^{i\theta_i}\right)
\end{equation}
with the gravitational weight normalization:
\begin{equation}
\Phi(x) = \arg\left(\frac{\sum_{i=1}^{d} g_i \lambda_i^2 e^{i\theta_i}}{\sum_{i=1}^{d} g_i \lambda_i^2}\right)
\end{equation}
where $g_i$ are the gravitational eigenvalues.
\end{theorem}

\begin{proof}
By Axiom A4, the phase operator extracts the "net phase direction" of an Elder space element. The weighting by $\lambda_i^2$ reflects the quadratic contribution of each component to the overall phase structure.

The gravitational normalization ensures that components with larger gravitational eigenvalues (stronger fields) contribute proportionally more to the overall phase, consistent with the hierarchical structure of Elder spaces.

The argument function $\arg(\cdot)$ projects the complex weighted sum onto the unit circle $\mathbb{S}^1$, giving the final phase value.
\end{proof}

\subsection{Elder Multiplication Detailed Analysis}

\begin{theorem}[Non-Commutative Multiplication Structure]
The Elder multiplication $x \star y$ for $x, y \in \elder{d}$ satisfies:
\begin{equation}
(x \star y)_k = \sum_{i,j=1}^{d} \lambda_i \mu_j e^{i(\theta_i + \phi_j)} \mathcal{M}_{ij}^k
\end{equation}
where $\mathcal{M}_{ij}^k$ are the structure constants satisfying:
\begin{align}
\mathcal{M}_{ij}^k &= \langle \elderstructure{i} \star \elderstructure{j}, \elderstructure{k} \rangle_E\\
\mathcal{M}_{ij}^k &= \mathcal{M}_{ji}^k \text{ when } \Phi(\elderstructure{i}) = \Phi(\elderstructure{j})\\
\sum_k \mathcal{M}_{ij}^k &= 1 \text{ (normalization)}
\end{align}
\end{theorem}

\begin{proof}
The multiplication structure follows from the basis element interactions defined by the phase orthogonality conditions in the Structural Elements theorem. The structure constants encode the geometric relationships between basis elements under the Elder multiplication.

The symmetry condition reflects the fact that elements with identical phases commute under multiplication, while the normalization ensures that multiplication preserves the overall "magnitude" of combinations.
\end{proof}