\section{Computational Complexity Analysis}

The computational properties of Elder space operations are fundamental to understanding their practical implementation and scalability.

\begin{theorem}[Operation Complexity]
For Elder spaces $\elder{d}$ with dimension $d$, the computational complexities are:
\begin{enumerate}
    \item \textbf{Addition $\oplus$}: $O(d + \log d)$ where $O(d)$ handles vector addition and $O(\log d)$ computes the phase combination
    \item \textbf{Scalar Multiplication $\odot$}: $O(d)$ for vector scaling plus $O(1)$ for phase scaling
    \item \textbf{Elder Multiplication $\star$}: $O(d^2)$ due to non-commutative interaction between components
    \item \textbf{Phase Operator $\Phi$}: $O(d)$ for computing the weighted phase combination
\end{enumerate}
\end{theorem}

\begin{proof}
\textbf{Addition Complexity}: The vector component requires $d$ scalar additions. The phase combination $\arg((z \cdot \overline{z} + \epsilon)\phi + (w \cdot \overline{w} + \epsilon)\psi)$ requires computing inner products in $O(d)$ time, then the argument computation in $O(\log d)$ time using CORDIC algorithms.

\textbf{Multiplication Complexity}: The non-commutative structure requires computing all pairwise interactions between basis elements, leading to $O(d^2)$ complexity. This can be reduced to $O(d \log d)$ using fast transform methods when basis elements have structured phase relationships.

\textbf{Space Complexity}: Elder spaces require $O(d)$ storage for the vector components plus $O(1)$ for the phase, giving total space complexity $O(d)$.
\end{proof}

\begin{corollary}[Scalability Properties]
Elder space operations scale polynomially in dimension, making them practical for moderate-dimensional applications while preserving the rich mathematical structure needed for hierarchical knowledge representation.
\end{corollary}

\section{Comparison with Alternative Structures}

\begin{theorem}[Structural Efficiency]
Elder spaces provide optimal complexity for phase-aware operations compared to alternative mathematical structures:
\begin{enumerate}
    \item \textbf{vs. Quaternions}: Elder spaces generalize quaternions to arbitrary dimension with the same $O(d^2)$ multiplication complexity
    \item \textbf{vs. Clifford Algebras}: Elder spaces achieve similar representational power with continuous phase modulation at comparable complexity
    \item \textbf{vs. Matrix Lie Groups}: Elder spaces avoid the $O(d^3)$ complexity of general matrix operations while preserving group-like structure
\end{enumerate}
\end{theorem}

This computational analysis demonstrates that Elder spaces achieve an optimal balance between mathematical expressiveness and computational efficiency, making them suitable for practical implementation in knowledge representation systems.