\begin{figure}[t]
\centering
\begin{tikzpicture}[scale=0.85, transform shape]
    % Define styles
    \tikzset{
        knowledge/.style={
            draw,
            fill=blue!15,
            circle,
            minimum size=2cm,
            align=center
        },
        composed/.style={
            draw,
            fill=green!15,
            circle,
            minimum size=2.5cm,
            align=center
        },
        emergent/.style={
            draw,
            fill=red!15,
            ellipse,
            minimum width=2.5cm,
            minimum height=1.8cm,
            text width=2.3cm,
            align=center
        },
        arrow/.style={
            ->,
            thick,
            >=latex
        },
        equation/.style={
            draw,
            fill=yellow!15,
            rounded corners,
            minimum width=5cm,
            minimum height=1cm,
            text width=4.8cm,
            align=center
        }
    }
    
    % Basic emergence illustration
    \begin{scope}[shift={(0,0)}]
        % Title
        \node[font=\bfseries] at (0,6) {Emergent Knowledge in Composition};
        
        % Knowledge elements
        \node[knowledge] (k1) at (-1.5,4) {$k_1$};
        \node[knowledge] (k2) at (1.5,4) {$k_2$};
        
        % Composition circle
        \begin{scope}
            \clip (0,2) circle (2cm);
            \fill[green!15] (-3,0) rectangle (3,4);
        \end{scope}
        \draw (0,2) circle (2cm);
        \node at (0,2) {$k_1 \oplus k_2$};
        
        % Emergent knowledge
        \node[emergent] (emerg) at (0,0) {Emergent\\Knowledge\\$k_{emergent}$};
        
        % Connect elements
        \draw[arrow] (k1) -- (0,2);
        \draw[arrow] (k2) -- (0,2);
        \draw[arrow] (0,1.5) -- (emerg);
        
        % Definition
        \node[equation] at (0,-1.5) {$k_{emergent} = k_{composite} \setminus \bigoplus_{i} k_i$};
    \end{scope}
    
    % Hierarchical emergence
    \begin{scope}[shift={(8,0)}]
        % Title
        \node[font=\bfseries] at (0,6) {Hierarchical Emergence};
        
        % Emergence at different levels
        \node[emergent, minimum width=1.8cm, minimum height=1.2cm, text width=1.6cm] (e_erudite) at (0,4) {Erudite\\Emergence\\$k_{emergent, E}$};
        
        \node[emergent, minimum width=2.2cm, minimum height=1.5cm, text width=2cm] (e_mentor) at (0,2.5) {Mentor\\Emergence\\$k_{emergent, M}$};
        
        \node[emergent, minimum width=2.5cm, minimum height=1.8cm, text width=2.3cm] (e_elder) at (0,0.5) {Elder\\Emergence\\$k_{emergent, El}$};
        
        % Information content relationship
        \draw[thick, <-] (e_erudite) -- (e_mentor) -- (e_elder);
        
        % Equation
        \node[equation] at (0,-1.5) {$I(k_{emergent, El}) > I(k_{emergent, M}) > I(k_{emergent, E})$};
    \end{scope}
    
    % Resonance-enhanced emergence
    \begin{scope}[shift={(0,-6)}]
        % Title
        \node[font=\bfseries] at (0,2) {Resonance-Enhanced Emergence};
        
        % Axes for plot
        \draw[->] (-0.5,0) -- (5,0) node[right] {Resonance Strength $r$};
        \draw[->] (0,-0.5) -- (0,4) node[above] {Emergent Information};
        
        % Curves for different n values
        \draw[domain=0:4.5, samples=100, smooth, variable=\x, red, thick] 
            plot ({\x}, {0.6*\x*ln(2)});
        \draw[domain=0:4.5, samples=100, smooth, variable=\x, blue, thick] 
            plot ({\x}, {0.6*\x*ln(4)});
        \draw[domain=0:4.5, samples=100, smooth, variable=\x, green!50!black, thick] 
            plot ({\x}, {0.6*\x*ln(8)});
        
        % Labels
        \node[red] at (4.5,1) {$n = 2$};
        \node[blue] at (4.5,2) {$n = 4$};
        \node[green!50!black] at (4.5,3) {$n = 8$};
        
        % Equation
        \node[equation] at (2.5,-1.5) {$I(k_{emergent}) \propto r \cdot \log(n)$};
    \end{scope}
    
    % Compositional generalization
    \begin{scope}[shift={(8,-6)}]
        % Title
        \node[font=\bfseries] at (0,2) {Compositional Generalization};
        
        % Axes for plot
        \draw[->] (-0.5,0) -- (5,0) node[right] {Number of Elements $n$};
        \draw[->] (0,-0.5) -- (0,4) node[above] {Generalization Error};
        
        % Element error curve
        \draw[domain=0:4.5, samples=100, smooth, variable=\x, blue, thick] 
            plot ({\x}, {1.6});
            
        % Composite error curve
        \draw[domain=0:4.5, samples=100, smooth, variable=\x, red, thick] 
            plot ({\x}, {1.6 + 1*exp(-0.5*\x) - 0.2*\x});
            
        % Threshold marker
        \draw[dashed] (2,0) -- (2,4);
        \node at (2,-0.3) {$n_0$};
        
        % Labels
        \node[blue] at (4.5,1.6) {$\min\{\epsilon_i\}$};
        \node[red] at (4.5,0.8) {$\epsilon_{composite}$};
        
        % Equation
        \node[equation] at (2.5,-1.5) {$\epsilon_{composite} < \min\{\epsilon_i\}$ when $n > n_0$};
    \end{scope}
    
\end{tikzpicture}
\caption{Emergence of knowledge in compositional systems. Top left: Emergent knowledge arises from composition as new patterns and relationships that aren't present in the individual elements, defined formally as $k_{emergent} = k_{composite} \setminus \bigoplus_{i} k_i$. Top right: The amount of emergent knowledge increases with hierarchical level, with $I(k_{emergent, El}) > I(k_{emergent, M}) > I(k_{emergent, E})$, reflecting how higher levels integrate across more diverse sources. Bottom left: Resonance-enhanced emergence scales as $I(k_{emergent}) \propto r \cdot \log(n)$, where $r$ is resonance strength and $n$ is the number of composed elements, with stronger resonance amplifying emergent patterns. Bottom right: Compositional generalization demonstrates a threshold effect where initially composition may increase error, but beyond a threshold $n_0$, the composite error becomes lower than the minimum individual error ($\epsilon_{composite} < \min\{\epsilon_i\}$ when $n > n_0$), creating a compositional advantage in generalization.}
\label{fig:emergent_knowledge}
\end{figure}