% Preamble file containing all package configurations and custom settings

% Basic packages
\usepackage[utf8]{inputenc}
\usepackage[T1]{fontenc}
\usepackage{textcomp}
\usepackage{amsmath,amssymb}

% Define common Unicode symbols
\DeclareUnicodeCharacter{2248}{\ensuremath{\approx}}
\usepackage{lmodern}  % Latin Modern fonts
\usepackage{microtype}
\usepackage{amsmath,amssymb,amsthm,amsfonts}
\usepackage{mathtools}
\usepackage{physics}
\usepackage{bm}  % Bold math symbols
\usepackage{graphicx}
\usepackage{xcolor}
\usepackage{hyperref}
\usepackage{bookmark}
% \usepackage[framemethod=tikz]{mdframed}
\usepackage{thmtools}
\usepackage{thm-restate}
\usepackage{enumitem}
\usepackage{multicol}
\usepackage{tabularx}
\usepackage{booktabs}
\usepackage{multirow}
\usepackage{caption}
\usepackage{subcaption}
\usepackage{tikz}
\usepackage{pgfplots}
\usepackage{algorithm2e}
\usepackage{algorithm}
\usepackage{algpseudocode}
\usepackage{algorithmicx}
\usepackage{listings}
\usepackage{fancyhdr}
\usepackage[]{geometry}
\usepackage[backend=biber, style=alphabetic]{biblatex}
\usepackage{csquotes}
\usepackage{makeidx}
\usepackage{minted}
\usepackage{titlesec}
\usepackage{titletoc}
\usepackage{etoolbox}
\usepackage{xparse}
\usepackage{marginnote}
\usepackage{sidenotes}
\usepackage[many]{tcolorbox}
\tcbuselibrary{listings,skins,breakable,theorems,raster,minted}

% Ensure all TikZ libraries are properly loaded
% Ensure listings environment is properly configured
\lstnewenvironment{lstlisting}[1][]
{\lstset{#1}}



% Define axis environment for compatibility
\let\axis\pgfplotsaxis
\let\endaxis\endpgfplotsaxis

% Define custom tcolorbox environments for different use cases
\newtcolorbox{examplebox}[1][]{
  enhanced,
  colback=ExampleGray!30!white,
  colframe=DarkGray,
  fonttitle=\bfseries,
  title={Example #1},
  breakable
}

\newtcolorbox{notebox}[1][]{
  enhanced,
  colback=LightGray!50!white,
  colframe=DarkGray,
  fonttitle=\bfseries,
  title={Note #1},
  breakable
}

\newtcolorbox{importantbox}[1][]{
  enhanced,
  colback=LemmaGreen!30!white,
  colframe=DarkGray,
  fonttitle=\bfseries,
  title={Important #1},
  breakable
}

\newtcolorbox{warningbox}[1][]{
  enhanced,
  colback=PropositionYellow!30!white,
  colframe=DarkGray,
  fonttitle=\bfseries,
  title={Warning #1},
  breakable
}

% Define tcolorbox styles for code listings
\newtcolorbox{codeblock}[1][]{
  colback=CodeBackground,
  colframe=DarkGray,
  arc=2mm,
  boxrule=0.5pt,
  left=5pt,
  right=5pt,
  top=5pt,
  bottom=5pt,
  listing only,
  listing options={
    basicstyle=\ttfamily\small,
    keywordstyle=\color{CodeKeyword},
    commentstyle=\color{CodeComment},
    stringstyle=\color{CodeString},
    breaklines=true,
    showstringspaces=false,
    tabsize=2,
    language=C++
  },
  title=#1,
  fonttitle=\bfseries
}

% Define observation environment
% Commented out to avoid conflicts with simpler definition below
% \newtcolorbox{observation}[1][]{%
%   enhanced,
%   colback=LightGray,
%   colframe=DarkGray,
%   fonttitle=\bfseries,
%   title={Observation #1},
%   breakable
% }

% Define axis environment if needed
\providecommand{\axis}{\begin{tikzpicture}\begin{axis}}
\providecommand{\endaxis}{\end{axis}\end{tikzpicture}}

% Define observation environment (simple version that doesn't depend on tcolorbox)
\newenvironment{observation}
{\begin{quote}\itshape\noindent\textbf{Observation:} }
{\end{quote}}

% Define colors
\definecolor{DarkSkyBlue}{RGB}{0, 51, 153}
\definecolor{LightGray}{RGB}{240, 240, 240}
\definecolor{DarkGray}{RGB}{64, 64, 64}
\definecolor{TheoremBlue}{RGB}{230, 236, 245}
\definecolor{LemmaGreen}{RGB}{230, 245, 230}
\definecolor{PropositionYellow}{RGB}{245, 245, 230}
\definecolor{DefinitionPurple}{RGB}{240, 230, 245}
\definecolor{ExampleGray}{RGB}{240, 240, 240}
\definecolor{CodeBackground}{RGB}{248, 248, 248}
\definecolor{CodeComment}{RGB}{0, 128, 0}
\definecolor{CodeKeyword}{RGB}{0, 0, 255}
\definecolor{CodeString}{RGB}{163, 21, 21}

% Page geometry
\geometry{
    paper=a4paper,
    inner=2.5cm,
    outer=2.5cm,
    top=2.5cm,
    bottom=2.5cm,
    marginparwidth=2cm,
    marginparsep=0.5cm,
    headsep=1cm,
    footskip=1cm
}

% Fix the headheight warning
\setlength{\headheight}{26pt}

% Set line spacing
\renewcommand{\baselinestretch}{1.3}

% Configure hyperref
\hypersetup{
    colorlinks=true,
    linkcolor=DarkSkyBlue,
    citecolor=DarkSkyBlue,
    urlcolor=DarkSkyBlue,
    pdfauthor={Author Name},
    pdftitle={Elder Theory},
    pdfsubject={Mathematical Text},
    pdfkeywords={mathematics, elder, realization}
}

% Configure listings for code
\lstset{
    basicstyle=\ttfamily\small,
    backgroundcolor=\color{CodeBackground},
    keywordstyle=\color{CodeKeyword},
    commentstyle=\color{CodeComment},
    stringstyle=\color{CodeString},
    numbers=left,
    numberstyle=\tiny\color{DarkGray},
    numbersep=5pt,
    frame=single,
    framesep=5pt,
    breaklines=true,
    breakatwhitespace=true,
    showstringspaces=false,
    tabsize=4
}

% GoLang language definition for listings
\lstdefinelanguage{golang}{
  keywords={break, default, func, interface, select, case, defer, go, map, struct, chan, else, goto, package, switch, const, fallthrough, if, range, type, continue, for, import, return, var},
  sensitive=true,
  morecomment=[l]{//},
  morecomment=[s]{/*}{*/},
  morestring=[b]",
  morestring=[b]',
  morestring=[b]`
}

% C++ language definition for listings
\lstdefinelanguage{C++}{
  keywords={alignas, alignof, and, and_eq, asm, atomic_cancel, atomic_commit, atomic_noexcept, auto, bitand, bitor, bool, break, case, catch, char, char8_t, char16_t, char32_t, class, compl, concept, const, consteval, constexpr, constinit, const_cast, continue, co_await, co_return, co_yield, decltype, default, delete, do, double, dynamic_cast, else, enum, explicit, export, extern, false, float, for, friend, goto, if, inline, int, long, mutable, namespace, new, noexcept, not, not_eq, nullptr, operator, or, or_eq, private, protected, public, reflexpr, register, reinterpret_cast, requires, return, short, signed, sizeof, static, static_assert, static_cast, struct, switch, synchronized, template, this, thread_local, throw, true, try, typedef, typeid, typename, union, unsigned, using, virtual, void, volatile, wchar_t, while, xor, xor_eq},
  sensitive=true,
  morecomment=[l]{//},
  morecomment=[s]{/*}{*/},
  morestring=[b]",
  morestring=[b]',
  otherkeywords={<, >, [, ], (, ), \{, \}, =, ==, !=, <=, >=, +, -, *, /, &, |, ^, ~, :, ;, %, ? },
  morekeywords={complex, float, ParallelFor, AllocateDeviceMemory, HermitianOuterProduct, ComplexMatrixMultiply, ComplexActivation, SumMatrices}
}

% Configure algorithm2e
\SetAlgorithmName{Algorithm}{Algorithm}{List of Algorithms}
\SetAlgoCaptionLayout{smallcaption}
\SetAlgoNoLine
\SetAlgoNoEnd
\SetArgSty{textnormal}
\SetDataSty{textsf}
\SetFuncSty{texttt}
\SetKwSty{textbf}
\SetKwInOut{Input}{Input}
\SetKwInOut{Output}{Output}

% TikZ and PGFPlots configuration
\pgfplotsset{compat=1.18}
\usetikzlibrary{
    arrows,
    automata,
    backgrounds,
    calc,
    decorations.pathmorphing,
    decorations.pathreplacing,
    fit,
    matrix,
    patterns,
    positioning,
    shapes.geometric,
    shapes.misc,
    arrows.meta,
    decorations.markings,
    plotmarks
}

% Enable pgfplots axis environment with all needed libraries
\usepgfplotslibrary{groupplots,fillbetween,colorbrewer,polar,statistics,dateplot,patchplots,external}

% Define a tikzpicture-like environment that wraps axis environment for convenience
\newenvironment{axisfigure}[1][]{%
\begin{tikzpicture}
\begin{axis}[#1]
}{%
\end{axis}
\end{tikzpicture}
}



% Bibliography file
\addbibresource{bibliography.bib}

% Set up index
\makeindex

% Load custom macro files
% Mathematical macros for "Elder, the Arcane Realization"

% Common sets
\newcommand{\N}{\mathbb{N}}  % Natural numbers
\newcommand{\Z}{\mathbb{Z}}  % Integers
\newcommand{\Q}{\mathbb{Q}}  % Rational numbers
\newcommand{\R}{\mathbb{R}}  % Real numbers
\newcommand{\C}{\mathbb{C}}  % Complex numbers
\newcommand{\F}{\mathbb{F}}  % Generic field

% Set operations
\newcommand{\union}{\cup}
\newcommand{\intersection}{\cap}
\newcommand{\compose}{\circ}
\newcommand{\tensor}{\otimes}
\newcommand{\bigtensor}{\bigotimes}

% Calculus
\newcommand{\deriv}[2]{\frac{d #1}{d #2}}
\newcommand{\pderiv}[2]{\frac{\partial #1}{\partial #2}}
\newcommand{\integral}[2]{\int_{#1}^{#2}}
\newcommand{\closed}[1]{\overline{#1}}
\newcommand{\open}[1]{\stackrel{\circ}{#1}}

% Linear algebra
\newcommand{\inner}[2]{\langle #1, #2 \rangle}
\newcommand{\norm}[1]{\left\lVert#1\right\rVert}
\newcommand{\abs}[1]{\left|#1\right|}
\newcommand{\transpose}{^{\mathsf{T}}}
\newcommand{\adj}{^{*}}
\newcommand{\tr}{\operatorname{tr}}
\newcommand{\rank}{\operatorname{rank}}
\newcommand{\nullity}{\operatorname{nullity}}
\newcommand{\im}{\operatorname{im}}
\newcommand{\vspan}{\operatorname{span}}

% Group theory
\newcommand{\group}[1]{\mathcal{#1}}
\newcommand{\subgroup}{\leqslant}
\newcommand{\normalsubgroup}{\trianglelefteq}
\newcommand{\quotient}[2]{#1/#2}
\newcommand{\conj}[2]{#1^{#2}}
\newcommand{\comm}[2]{[#1,#2]}
\newcommand{\commutator}[2]{[#1,#2]}

% Category theory
\newcommand{\cat}[1]{\mathbf{#1}}
\newcommand{\Fun}{\operatorname{Fun}}
\newcommand{\Hom}{\operatorname{Hom}}
\newcommand{\End}{\operatorname{End}}
\newcommand{\Aut}{\operatorname{Aut}}
\newcommand{\id}{\mathrm{id}}
\newcommand{\iso}{\cong}
\newcommand{\too}{\longrightarrow}
\newcommand{\functorial}[1]{#1^{\bullet}}

% Topology
\newcommand{\closure}[1]{\overline{#1}}
\newcommand{\interior}[1]{\mathring{#1}}
\newcommand{\boundary}[1]{\partial #1}
\newcommand{\connected}{\text{connected}}
\newcommand{\compact}{\text{compact}}
\newcommand{\covering}{\text{covering}}

% Analysis
\newcommand{\limsup}{\varlimsup}
\newcommand{\liminf}{\varliminf}
\newcommand{\tendsto}{\rightarrow}
\newcommand{\converge}{\xrightarrow{\text{conv.}}}
\newcommand{\weakconverge}{\xrightarrow{\text{w.}}}
\newcommand{\uniformconverge}{\xrightarrow{\text{unif.}}}

% Probability
\newcommand{\Prob}{\mathbb{P}}
\newcommand{\Expectation}{\mathbb{E}}
\newcommand{\Variance}{\operatorname{Var}}
\newcommand{\Cov}{\operatorname{Cov}}
\newcommand{\distribution}{\sim}

% Logic
\newcommand{\implies}{\Rightarrow}
\newcommand{\iff}{\Leftrightarrow}
\newcommand{\notimplies}{\not\Rightarrow}
\newcommand{\notiff}{\not\Leftrightarrow}
\newcommand{\forall}{\forall}
\newcommand{\exists}{\exists}
\newcommand{\existsunique}{\exists!}

% Specific to the book 
\newcommand{\arcane}[1]{\mathfrak{A}_{#1}}
\newcommand{\elder}[1]{\mathcal{E}_{#1}}
\newcommand{\realization}[1]{\mathscr{R}(#1)}
\newcommand{\arcanesequence}[1]{\{A_{#1}\}}
\newcommand{\eldestate}{\mathbf{\Psi}}

% Common operators
\DeclareMathOperator{\lcm}{lcm}
\DeclareMathOperator{\gcd}{gcd}
\DeclareMathOperator{\ord}{ord}
\DeclareMathOperator{\sgn}{sgn}
\DeclareMathOperator{\diag}{diag}
\DeclareMathOperator{\char}{char}
\DeclareMathOperator{\deg}{deg}
\DeclareMathOperator{\supp}{supp}

% Theorem and Environment Styling for Elder Framework
% This file defines the visual appearance of theorems, definitions, examples, etc.

% Load required packages
\usepackage{amsthm}
\usepackage{thmtools}
\usepackage{tcolorbox}
\tcbuselibrary{theorems}

% Define color scheme
\definecolor{TheoremBlue}{RGB}{230, 236, 245}
\definecolor{LemmaGreen}{RGB}{230, 245, 230}
\definecolor{PropositionYellow}{RGB}{245, 245, 230}
\definecolor{DefinitionPurple}{RGB}{240, 230, 245}
\definecolor{ExampleGray}{RGB}{240, 240, 240}
\definecolor{NotationColor}{RGB}{255, 240, 245}

% Create theorem styles with tcolorbox
\tcbset{
    theoremstyle/.style={
        enhanced,
        breakable,
        colback=#1!10!white,
        colframe=#1!85!black,
        fonttitle=\bfseries,
        coltitle=black,
        colbacktitle=#1!20!white,
        attach title to upper=\par\smallskip,
        top=2mm,
        bottom=2mm,
        left=4mm,
        right=4mm,
        arc=1mm,
        before skip=8pt,
        after skip=8pt
    }
}

% Define theorem-like environments
\declaretheorem[style=tcbtheorem, 
                tcb={theoremstyle=TheoremBlue}, 
                name=Theorem,
                numberwithin=chapter]{theorem}

\declaretheorem[style=tcbtheorem, 
                tcb={theoremstyle=LemmaGreen}, 
                name=Lemma,
                sibling=theorem]{lemma}

\declaretheorem[style=tcbtheorem, 
                tcb={theoremstyle=PropositionYellow}, 
                name=Proposition,
                sibling=theorem]{proposition}

\declaretheorem[style=tcbtheorem, 
                tcb={theoremstyle=TheoremBlue}, 
                name=Corollary,
                sibling=theorem]{corollary}

\declaretheorem[style=tcbtheorem, 
                tcb={theoremstyle=DefinitionPurple}, 
                name=Definition,
                numberwithin=chapter]{definition}

\declaretheorem[style=tcbtheorem, 
                tcb={theoremstyle=ExampleGray}, 
                name=Example,
                numberwithin=chapter]{example}

\declaretheorem[style=tcbtheorem, 
                tcb={theoremstyle=ExampleGray}, 
                name=Remark,
                numberwithin=chapter]{remark}

% Non-numbered environments
\declaretheorem[style=tcbtheorem, 
                tcb={theoremstyle=NotationColor}, 
                name=Notation,
                numbered=no]{notation}

\declaretheorem[style=tcbtheorem, 
                tcb={theoremstyle=ExampleGray}, 
                name=Algorithm,
                numberwithin=chapter]{algorithm}

% Optional: proof environment styling
\renewenvironment{proof}{
  \pushQED{\qed}
  \normalfont \topsep6\p@\@plus6\p@\relax
  \trivlist\item[\hskip\labelsep
  \itshape\sffamily{Proof.}]\mbox{}\newline
}{
  \popQED\endtrivlist\@endpefalse
}

% Custom "Elder Theorem" environment for especially important results
\declaretheorem[style=tcbtheorem, 
                tcb={theoremstyle=TheoremBlue, 
                     colframe=DarkSkyBlue!90!black,
                     leftrule=3mm}, 
                name={Elder Theorem},
                numberwithin=chapter]{eldertheorem}

% List of theorems configuration
\declaretheoremstyle[
  spaceabove=6pt, 
  spacebelow=6pt,
  notefont=\bfseries, 
  notebraces={[}{]},
  bodyfont=\itshape,
]{thmstyle}

% Table of theorems setup
\newcommand{\listofeldertheorems}{
  \chapter*{List of Principal Results}
  \markboth{LIST OF PRINCIPAL RESULTS}{LIST OF PRINCIPAL RESULTS}
  \addcontentsline{toc}{chapter}{List of Principal Results}
  \begin{description}
    \item[Theorems] provide the main mathematical results in the Elder framework.
    \item[Lemmas] are supporting propositions used to build toward major theorems.
    \item[Propositions] state important facts with complete proofs.
    \item[Definitions] establish the precise meaning of mathematical concepts.
    \item[Examples] demonstrate the application of theoretical concepts.
  \end{description}
  \listoftheorems[ignoreall,show={theorem,eldertheorem}]
}

% Theorem reference formatting
\renewcommand{\thetheoremrefs}{\thechapter.\arabic{theorem}}
\renewcommand{\thedefinitionrefs}{\thechapter.\arabic{definition}}

% Load custom style files
% Chapter style definitions for "Elder, the Arcane Realization"

% Create thumbnail commands
\newcommand{\chapterthumbnail}[1]{%
    \begingroup
    \begin{tikzpicture}[remember picture, overlay]
        \node[anchor=north east, 
              inner sep=0pt, 
              outer sep=0pt] 
             at ([xshift=-25pt, yshift=-50pt]current page.north east) 
             {\includegraphics[width=1.5cm]{#1}};
    \end{tikzpicture}
    \endgroup
}

% Empty thumbnail for chapters without specific thumbnails
\newcommand{\nochapterthumbnail}{%
    \begingroup
    \begin{tikzpicture}[remember picture, overlay]
        \node[anchor=north east, 
              inner sep=0pt, 
              outer sep=0pt,
              minimum width=1.5cm,
              minimum height=1.5cm] 
             at ([xshift=-25pt, yshift=-50pt]current page.north east) 
             {};
    \end{tikzpicture}
    \endgroup
}

% Chapter heading style
\titleformat{\chapter}[display]
    {\normalfont\huge\bfseries}
    {\filleft\begin{minipage}{5cm}
    \flushright{\fontsize{80}{80}\color{DarkSkyBlue}\selectfont\thechapter}
    \end{minipage}}
    {20pt}
    {\titlerule\vspace{10pt}\filright}
    [\vspace{10pt}]

% Chapter style in table of contents
\titlecontents{chapter}
    [1.5em] % left margin
    {\addvspace{1.0em}\large\bfseries} % above code
    {\contentslabel{1.5em}\color{DarkSkyBlue}} % numbered format
    {\hspace*{-1.5em}} % unnumbered format
    {\hfill\contentspage} % filler-page format
    [\addvspace{0.5em}] % below code

% Improved part style
\titleformat{\part}[display]
    {\centering\normalfont\Huge\bfseries}
    {\color{DarkSkyBlue}\fontsize{100}{100}\selectfont\thepart}
    {20pt}
    {\vspace{20pt}\color{DarkSkyBlue}}

% Part style in table of contents
\titlecontents{part}
    [0em] % left margin
    {\addvspace{2.0em}\Large\bfseries\centering} % above code
    {\color{DarkSkyBlue}PART~\thecontentslabel\\[0.5em]} % numbered format
    {} % unnumbered format
    {} % filler-page format
    [\addvspace{1.0em}] % below code
    
% Create a custom command for unit headings (between parts and chapters)
\newcounter{unit}[part]
\newcommand{\unit}[1]{%
    \stepcounter{unit}%
    \cleardoublepage
    \thispagestyle{plain}%
    \begin{center}%
        \vspace*{2cm}%
        {\color{DarkSkyBlue}\rule{\textwidth}{1pt}}\\[10pt]%
        {\Large\bfseries UNIT \theunit}\\[5pt]%
        {\LARGE\bfseries #1}\\[10pt]%
        {\color{DarkSkyBlue}\rule{\textwidth}{1pt}}%
        \vspace{2cm}%
    \end{center}%
    \addcontentsline{toc}{section}{Unit \theunit: #1}%
}

% Chapter summary environment
\newenvironment{chaptersummary}
    {\begin{mdframed}[
        linewidth=0.5pt,
        linecolor=DarkSkyBlue,
        backgroundcolor=DarkSkyBlue!5,
        innertopmargin=10pt,
        innerbottommargin=10pt,
        innerrightmargin=10pt,
        innerleftmargin=10pt,
        skipabove=15pt,
        skipbelow=15pt
    ]
    \textbf{Chapter Summary:}\\}
    {\end{mdframed}}

% Problem set environment at the end of chapters
\newenvironment{problemset}[1][\arabic{chapter}]
    {\cleardoublepage
     \section*{Problems for Chapter #1}
     \addcontentsline{toc}{section}{Problems for Chapter #1}
     \begin{enumerate}[label=\textbf{\arabic{chapter}.\arabic*.}]}
    {\end{enumerate}}

% Historical note environment
\newenvironment{historicalnote}[1]
    {\begin{mdframed}[
        linewidth=0.5pt,
        linecolor=DarkGray,
        backgroundcolor=LightGray!30,
        innertopmargin=10pt,
        innerbottommargin=10pt,
        innerrightmargin=10pt,
        innerleftmargin=10pt,
        skipabove=15pt,
        skipbelow=15pt
    ]
    \textbf{Historical Note: #1}\\}
    {\end{mdframed}}

% Advanced topic environment
\newenvironment{advancedtopic}[1]
    {\begin{mdframed}[
        linewidth=0.5pt,
        linecolor=DarkSkyBlue,
        backgroundcolor=DarkSkyBlue!10,
        innertopmargin=10pt,
        innerbottommargin=10pt,
        innerrightmargin=10pt,
        innerleftmargin=10pt,
        skipabove=15pt,
        skipbelow=15pt
    ]
    \textbf{Advanced Topic: #1}\\}
    {\end{mdframed}}

% Layout style definitions for "Elder, the Arcane Realization"

% Two-column environment with custom settings
\NewDocumentEnvironment{twocolumnlayout}{}{
    \begin{multicols}{2}
    \setlength{\columnsep}{20pt}
    \setlength{\columnseprule}{0pt}
}{
    \end{multicols}
}

% Custom margin note styling
\renewcommand{\marginnote}[1]{%
  \marginpar{%
    \raggedright\footnotesize\color{DarkGray}#1%
  }%
}

% Side note environment
\newcommand{\eldernote}[1]{%
  \marginpar{%
    \begin{tcolorbox}[
      colback=LightGray!30,
      colframe=DarkGray!30,
      boxrule=0.5pt,
      arc=0pt,
      left=5pt,
      right=5pt,
      top=5pt,
      bottom=5pt,
      boxsep=2pt,
      fontupper=\footnotesize
    ]
    #1
    \end{tcolorbox}
  }%
}

% Margin theorem style
\newenvironment{margintheorem}[2][\empty]{%
  \marginpar{%
    \begin{tcolorbox}[
      colback=TheoremBlue!20,
      colframe=TheoremBlue!80,
      boxrule=0.5pt,
      arc=0pt,
      left=3pt,
      right=3pt,
      top=3pt,
      bottom=3pt,
      boxsep=2pt,
      fontupper=\footnotesize,
      title={\footnotesize\textbf{Theorem\ifx\empty#1\else~#1\fi}}
    ]
    #2
    \end{tcolorbox}
  }%
}{\par}

% Highlighted text box
\newcommand{\highlight}[1]{%
  \begin{tcolorbox}[
    enhanced,
    breakable,
    colback=DarkSkyBlue!5,
    colframe=DarkSkyBlue!40,
    boxrule=0.5pt,
    arc=0pt,
    left=10pt,
    right=10pt,
    top=8pt,
    bottom=8pt,
    boxsep=5pt
  ]
  #1
  \end{tcolorbox}
}

% Key concept box
\newcommand{\keyconcept}[2]{%
  \begin{tcolorbox}[
    enhanced,
    breakable,
    colback=LemmaGreen!10,
    colframe=LemmaGreen!50,
    boxrule=0.5pt,
    arc=0pt,
    left=10pt,
    right=10pt,
    top=8pt,
    bottom=8pt,
    boxsep=5pt,
    title={\textbf{Key Concept: #1}}
  ]
  #2
  \end{tcolorbox}
}

% Figure environment with custom styling
\newenvironment{elegantfigure}[3][htbp]{%
  \begin{figure}[#1]
    \centering
    #2
    \caption{#3}
}{%
  \end{figure}
}

% Table environment with custom styling
\newenvironment{eleganttable}[3][htbp]{%
  \begin{table}[#1]
    \centering
    \caption{#3}
    \begin{tabular}{#2}
}{%
    \end{tabular}
  \end{table}
}

% Full width environment (breaks out of two column format)
\newenvironment{fullwidth}{%
  \end{multicols}
}{%
  \begin{multicols}{2}
}

% Double-column equation environment
\NewDocumentEnvironment{wideequation}{}{
  \end{multicols}
  \begin{equation}
}{
  \end{equation}
  \begin{multicols}{2}
}

% Custom chapter header and footer style
\fancypagestyle{chapterstyle}{%
  \fancyhf{}
  \fancyhead[LE]{\small\textit{\leftmark}}
  \fancyhead[RO]{\small\textit{\rightmark}}
  \fancyfoot[LE,RO]{\thepage}
  \renewcommand{\headrulewidth}{0.5pt}
  \renewcommand{\footrulewidth}{0pt}
}

% Custom quote environment
\newenvironment{elegantquote}[1][]{%
  \begin{quote}
  \itshape
  \color{DarkGray}
  \def\quoteauthor{#1}
}{%
  \ifx\quoteauthor\empty\else
  \par\hfill--- \textsc{\quoteauthor}
  \fi
  \end{quote}
}

% Custom part divider page
\newcommand{\partdivider}[2]{%
  \cleardoublepage
  \thispagestyle{empty}
  \begin{tikzpicture}[remember picture, overlay]
    \fill[DarkSkyBlue!20] (current page.north west) rectangle (current page.south east);
    \node[anchor=center, 
          inner sep=0pt, 
          outer sep=0pt] 
         at (current page.center) 
         {\begin{minipage}{\textwidth}
            \centering
            {\Huge\bfseries\textcolor{DarkSkyBlue}{PART \thepart}}\\[1cm]
            {\huge\bfseries #1}\\[0.5cm]
            {\Large #2}
          \end{minipage}};
  \end{tikzpicture}
  \clearpage
}

% Advanced topic environment
\newenvironment{advancedtopic}[1]{%
  \bigskip\noindent
  \begin{quote}
  \textbf{Advanced Topic: #1}\\
}{%
  \end{quote}
  \bigskip
}

% Chapter summary environment
\newenvironment{chaptersummary}{%
  \bigskip\noindent
  \begin{quote}
  \textbf{Chapter Summary}\\
}{%
  \end{quote}
  \bigskip
}

% Historical note environment
\newenvironment{historicalnote}[1]{%
  \bigskip\noindent
  \begin{quote}
  \textbf{Historical Note: #1}\\
}{%
  \end{quote}
  \bigskip
}

% Problem set environment
\newenvironment{problemset}{%
  \begin{enumerate}
}{%
  \end{enumerate}
}


% Define the two-column layout
\newenvironment{twocolumns}
{\begin{multicols}{2}}
{\end{multicols}}

% Create custom chapter display with margin numbers
\titleformat{\chapter}[display]
{\normalfont\huge\bfseries}
{\filleft\begin{minipage}{5cm}
\flushright{\fontsize{80}{80}\color{DarkSkyBlue}\selectfont\thechapter}
\end{minipage}}
{20pt}
{\titlerule\vspace{10pt}\filright}
[\vspace{10pt}]

% Custom section styling - ensure titles fit within column width in two-column layout
\titleformat{\section}
{\normalfont\Large\bfseries\color{DarkSkyBlue}}
{\thesection}{1em}{}[\vspace{0.5em}]

\titleformat{\subsection}
{\normalfont\large\bfseries}
{\thesubsection}{1em}{}[\vspace{0.3em}]

\titleformat{\subsubsection}
{\normalfont\normalsize\bfseries}
{\thesubsubsection}{1em}{}[\vspace{0.2em}]

% Adjust section title spacing to prevent overhang in two-column layout
\titlespacing*{\section}{0pt}{3.5ex plus 1ex minus .2ex}{2.3ex plus .2ex}
\titlespacing*{\subsection}{0pt}{3.25ex plus 1ex minus .2ex}{1.5ex plus .2ex}
\titlespacing*{\subsubsection}{0pt}{3.25ex plus 1ex minus .2ex}{1.5ex plus .2ex}

% Header and footer setup
\pagestyle{fancy}
\fancyhf{}
\fancyhead[LE,RO]{\thepage}
\fancyhead[RE]{\textit{\leftmark}}
\fancyhead[LO]{\textit{\rightmark}}
\renewcommand{\headrulewidth}{0.5pt}
\renewcommand{\footrulewidth}{0pt}

% Special pages style
\fancypagestyle{plain}{
    \fancyhf{}
    \fancyfoot[C]{\thepage}
    \renewcommand{\headrulewidth}{0pt}
    \renewcommand{\footrulewidth}{0pt}
}

% Custom list of theorems
\newcommand{\listofelderthms}{
    \chapter*{List of Theorems}
    \markboth{LIST OF THEOREMS}{LIST OF THEOREMS}
    \addcontentsline{toc}{chapter}{List of Theorems}
    \begingroup
    \let\clearpage\relax
    \listoftheorems[ignoreall,show={theorem}]
    \endgroup
}
