% Layout style definitions for "Elder, the Arcane Realization"

% Two-column environment with custom settings
\NewDocumentEnvironment{twocolumnlayout}{}{
    \begin{multicols}{2}
    \setlength{\columnsep}{20pt}
    \setlength{\columnseprule}{0pt}
}{
    \end{multicols}
}

% Custom margin note styling
\renewcommand{\marginnote}[1]{%
  \marginpar{%
    \raggedright\footnotesize\color{DarkGray}#1%
  }%
}

% Side note environment
\newcommand{\eldernote}[1]{%
  \marginpar{%
    \begin{tcolorbox}[
      colback=LightGray!30,
      colframe=DarkGray!30,
      boxrule=0.5pt,
      arc=0pt,
      left=5pt,
      right=5pt,
      top=5pt,
      bottom=5pt,
      boxsep=2pt,
      fontupper=\footnotesize
    ]
    #1
    \end{tcolorbox}
  }%
}

% Margin theorem style
\newenvironment{margintheorem}[2][\empty]{%
  \marginpar{%
    \begin{tcolorbox}[
      colback=TheoremBlue!20,
      colframe=TheoremBlue!80,
      boxrule=0.5pt,
      arc=0pt,
      left=3pt,
      right=3pt,
      top=3pt,
      bottom=3pt,
      boxsep=2pt,
      fontupper=\footnotesize,
      title={\footnotesize\textbf{Theorem\ifx\empty#1\else~#1\fi}}
    ]
    #2
    \end{tcolorbox}
  }%
}{\par}

% Highlighted text box
\newcommand{\highlight}[1]{%
  \begin{tcolorbox}[
    enhanced,
    breakable,
    colback=DarkSkyBlue!5,
    colframe=DarkSkyBlue!40,
    boxrule=0.5pt,
    arc=0pt,
    left=10pt,
    right=10pt,
    top=8pt,
    bottom=8pt,
    boxsep=5pt
  ]
  #1
  \end{tcolorbox}
}

% Key concept box
\newcommand{\keyconcept}[2]{%
  \begin{tcolorbox}[
    enhanced,
    breakable,
    colback=LemmaGreen!10,
    colframe=LemmaGreen!50,
    boxrule=0.5pt,
    arc=0pt,
    left=10pt,
    right=10pt,
    top=8pt,
    bottom=8pt,
    boxsep=5pt,
    title={\textbf{Key Concept: #1}}
  ]
  #2
  \end{tcolorbox}
}

% Figure environment with custom styling
\newenvironment{elegantfigure}[3][htbp]{%
  \begin{figure}[#1]
    \centering
    #2
    \caption{#3}
}{%
  \end{figure}
}

% Table environment with custom styling
\newenvironment{eleganttable}[3][htbp]{%
  \begin{table}[#1]
    \centering
    \caption{#3}
    \begin{tabular}{#2}
}{%
    \end{tabular}
  \end{table}
}

% Full width environment (breaks out of two column format)
\newenvironment{fullwidth}{%
  \end{multicols}
}{%
  \begin{multicols}{2}
}

% Double-column equation environment
\NewDocumentEnvironment{wideequation}{}{
  \end{multicols}
  \begin{equation}
}{
  \end{equation}
  \begin{multicols}{2}
}

% Custom chapter header and footer style
\fancypagestyle{chapterstyle}{%
  \fancyhf{}
  \fancyhead[LE]{\small\textit{\leftmark}}
  \fancyhead[RO]{\small\textit{\rightmark}}
  \fancyfoot[LE,RO]{\thepage}
  \renewcommand{\headrulewidth}{0.5pt}
  \renewcommand{\footrulewidth}{0pt}
}

% Custom quote environment
\newenvironment{elegantquote}[1][]{%
  \begin{quote}
  \itshape
  \color{DarkGray}
  \def\quoteauthor{#1}
}{%
  \ifx\quoteauthor\empty\else
  \par\hfill--- \textsc{\quoteauthor}
  \fi
  \end{quote}
}

% Custom part divider page
\newcommand{\partdivider}[2]{%
  \cleardoublepage
  \thispagestyle{empty}
  \begin{tikzpicture}[remember picture, overlay]
    \fill[DarkSkyBlue!20] (current page.north west) rectangle (current page.south east);
    \node[anchor=center, 
          inner sep=0pt, 
          outer sep=0pt] 
         at (current page.center) 
         {\begin{minipage}{\textwidth}
            \centering
            {\Huge\bfseries\textcolor{DarkSkyBlue}{PART \thepart}}\\[1cm]
            {\huge\bfseries #1}\\[0.5cm]
            {\Large #2}
          \end{minipage}};
  \end{tikzpicture}
  \clearpage
}

% Advanced topic environment
\newenvironment{advancedtopic}[1]{%
  \bigskip\noindent
  \begin{quote}
  \textbf{Advanced Topic: #1}\\
}{%
  \end{quote}
  \bigskip
}

% Chapter summary environment
\newenvironment{chaptersummary}{%
  \bigskip\noindent
  \begin{quote}
  \textbf{Chapter Summary}\\
}{%
  \end{quote}
  \bigskip
}

% Historical note environment
\newenvironment{historicalnote}[1]{%
  \bigskip\noindent
  \begin{quote}
  \textbf{Historical Note: #1}\\
}{%
  \end{quote}
  \bigskip
}

% Problem set environment
\newenvironment{problemset}{%
  \begin{enumerate}
}{%
  \end{enumerate}
}
