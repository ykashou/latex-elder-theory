% Chapter style definitions for "Elder, the Arcane Realization"

% Create thumbnail commands
\newcommand{\chapterthumbnail}[1]{%
    \begingroup
    \begin{tikzpicture}[remember picture, overlay]
        \node[anchor=north east, 
              inner sep=0pt, 
              outer sep=0pt] 
             at ([xshift=-25pt, yshift=-50pt]current page.north east) 
             {\includegraphics[width=1.5cm]{#1}};
    \end{tikzpicture}
    \endgroup
}

% Empty thumbnail for chapters without specific thumbnails
\newcommand{\nochapterthumbnail}{%
    \begingroup
    \begin{tikzpicture}[remember picture, overlay]
        \node[anchor=north east, 
              inner sep=0pt, 
              outer sep=0pt,
              minimum width=1.5cm,
              minimum height=1.5cm] 
             at ([xshift=-25pt, yshift=-50pt]current page.north east) 
             {};
    \end{tikzpicture}
    \endgroup
}

% Chapter heading style
\titleformat{\chapter}[display]
    {\normalfont\huge\bfseries}
    {\filleft\begin{minipage}{5cm}
    \flushright{\fontsize{80}{80}\color{DarkSkyBlue}\selectfont\thechapter}
    \end{minipage}}
    {20pt}
    {\titlerule\vspace{10pt}\filright}
    [\vspace{10pt}]

% Chapter style in table of contents
\titlecontents{chapter}
    [1.5em] % left margin
    {\addvspace{1.0em}\large\bfseries} % above code
    {\contentslabel{1.5em}\color{DarkSkyBlue}} % numbered format
    {\hspace*{-1.5em}} % unnumbered format
    {\hfill\contentspage} % filler-page format
    [\addvspace{0.5em}] % below code

% Improved part style
\titleformat{\part}[display]
    {\centering\normalfont\Huge\bfseries}
    {\color{DarkSkyBlue}\fontsize{100}{100}\selectfont\thepart}
    {20pt}
    {\vspace{20pt}\color{DarkSkyBlue}}

% Part style in table of contents
\titlecontents{part}
    [0em] % left margin
    {\addvspace{2.0em}\Large\bfseries\centering} % above code
    {\color{DarkSkyBlue}PART~\thecontentslabel\\[0.5em]} % numbered format
    {} % unnumbered format
    {} % filler-page format
    [\addvspace{1.0em}] % below code
    
% Create a custom command for unit headings (between parts and chapters)
\newcounter{unit}[part]
\newcommand{\unit}[1]{%
    \stepcounter{unit}%
    \cleardoublepage
    \thispagestyle{plain}%
    \begin{center}%
        \vspace*{2cm}%
        {\color{DarkSkyBlue}\rule{\textwidth}{1pt}}\\[10pt]%
        {\Large\bfseries UNIT \theunit}\\[5pt]%
        {\LARGE\bfseries #1}\\[10pt]%
        {\color{DarkSkyBlue}\rule{\textwidth}{1pt}}%
        \vspace{2cm}%
    \end{center}%
    \addcontentsline{toc}{section}{Unit \theunit: #1}%
}

% Chapter summary environment
\newenvironment{chaptersummary}
    {\begin{mdframed}[
        linewidth=0.5pt,
        linecolor=DarkSkyBlue,
        backgroundcolor=DarkSkyBlue!5,
        innertopmargin=10pt,
        innerbottommargin=10pt,
        innerrightmargin=10pt,
        innerleftmargin=10pt,
        skipabove=15pt,
        skipbelow=15pt
    ]
    \textbf{Chapter Summary:}\\}
    {\end{mdframed}}

% Problem set environment at the end of chapters
\newenvironment{problemset}[1][\arabic{chapter}]
    {\cleardoublepage
     \section*{Problems for Chapter #1}
     \addcontentsline{toc}{section}{Problems for Chapter #1}
     \begin{enumerate}[label=\textbf{\arabic{chapter}.\arabic*.}]}
    {\end{enumerate}}

% Historical note environment
\newenvironment{historicalnote}[1]
    {\begin{mdframed}[
        linewidth=0.5pt,
        linecolor=DarkGray,
        backgroundcolor=LightGray!30,
        innertopmargin=10pt,
        innerbottommargin=10pt,
        innerrightmargin=10pt,
        innerleftmargin=10pt,
        skipabove=15pt,
        skipbelow=15pt
    ]
    \textbf{Historical Note: #1}\\}
    {\end{mdframed}}

% Advanced topic environment
\newenvironment{advancedtopic}[1]
    {\begin{mdframed}[
        linewidth=0.5pt,
        linecolor=DarkSkyBlue,
        backgroundcolor=DarkSkyBlue!10,
        innertopmargin=10pt,
        innerbottommargin=10pt,
        innerrightmargin=10pt,
        innerleftmargin=10pt,
        skipabove=15pt,
        skipbelow=15pt
    ]
    \textbf{Advanced Topic: #1}\\}
    {\end{mdframed}}
