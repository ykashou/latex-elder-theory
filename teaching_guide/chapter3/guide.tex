% Chapter 3 Teaching Guide
% Unified Parameter Space
% TO BE DEVELOPED

\documentclass[12pt,a4paper]{article}
\usepackage{amsmath,amsthm,amssymb,amsfonts}
\usepackage{tcolorbox}
\usepackage{enumitem}
\usepackage{hyperref}
\usepackage{geometry}

\geometry{margin=1in}

\title{\textbf{Teaching Guide for Chapter 3:}\\
\Large{Unified Parameter Space}\\
\large{Instructor Resources}}

\author{Elder Theory Teaching Resources}
\date{October 2025}

\begin{document}

\maketitle

\begin{abstract}
This teaching guide provides instructors with comprehensive resources for teaching Chapter 3 of Elder Theory. It includes learning objectives, lecture notes, discussion prompts, assessment rubrics, and complete solutions to all exercises.

\textbf{Status}: 🚧 Under Development - Content to be added
\end{abstract}

\tableofcontents
\newpage

\section{Overview}

\subsection{Chapter Summary}
Chapter 3 introduces the unified parameter space framework, including:
\begin{itemize}
    \item Parameter space definitions
    \item Coordinate systems
    \item Gravitational parameters
    \item Phase relationships
    \item Transformations and mappings
\end{itemize}

\subsection{Learning Objectives}

By the end of this chapter, students should be able to:
\begin{enumerate}
    \item Define the unified parameter space
    \item Work with parameter space coordinates
    \item Understand gravitational parameter relationships
    \item Apply transformations in parameter space
    \item Connect parameter space to Elder space structure
\end{enumerate}

\subsection{Prerequisites}

Students should have completed:
\begin{itemize}
    \item Chapter 1: Introduction to Elder Spaces
    \item Chapter 2: Introduction to Elder Topology
    \item Understanding of coordinate systems
\end{itemize}

\section{Lecture Notes}

\textit{To be developed}

\section{Discussion Prompts}

\textit{To be developed}

\section{Assessment Rubrics}

\textit{To be developed}

\section{Solutions Manual}

\textit{To be developed}

\section{Common Student Difficulties}

\textit{To be developed}

\end{document}

