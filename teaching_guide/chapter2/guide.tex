% Chapter 2 Teaching Guide
% Introduction to Elder Topology
% TO BE DEVELOPED

\documentclass[12pt,a4paper]{article}
\usepackage{amsmath,amsthm,amssymb,amsfonts}
\usepackage{tcolorbox}
\usepackage{enumitem}
\usepackage{hyperref}
\usepackage{geometry}

\geometry{margin=1in}

\title{\textbf{Teaching Guide for Chapter 2:}\\
\Large{Introduction to Elder Topology}\\
\large{Instructor Resources}}

\author{Elder Theory Teaching Resources}
\date{October 2025}

\begin{document}

\maketitle

\begin{abstract}
This teaching guide provides instructors with comprehensive resources for teaching Chapter 2 of Elder Theory. It includes learning objectives, lecture notes, discussion prompts, assessment rubrics, and complete solutions to all exercises.

\textbf{Status}: 🚧 Under Development - Content to be added
\end{abstract}

\tableofcontents
\newpage

\section{Overview}

\subsection{Chapter Summary}
Chapter 2 introduces topological concepts in Elder spaces, including:
\begin{itemize}
    \item Topological structure of Elder spaces
    \item Open and closed sets
    \item Continuity and convergence
    \item Gravitational stratification
    \item Phase operator continuity
\end{itemize}

\subsection{Learning Objectives}

By the end of this chapter, students should be able to:
\begin{enumerate}
    \item Define the topological structure on an Elder space
    \item Identify open and closed sets in Elder topology
    \item Verify continuity of mappings between Elder spaces
    \item Understand gravitational stratification
    \item Apply topological concepts to phase operators
\end{enumerate}

\subsection{Prerequisites}

Students should have completed:
\begin{itemize}
    \item Chapter 1: Introduction to Elder Spaces
    \item Basic topology (metric spaces, open sets)
    \item Understanding of continuity
\end{itemize}

\section{Lecture Notes}

\textit{To be developed}

\section{Discussion Prompts}

\textit{To be developed}

\section{Assessment Rubrics}

\textit{To be developed}

\section{Solutions Manual}

\textit{To be developed}

\section{Common Student Difficulties}

\textit{To be developed}

\end{document}

