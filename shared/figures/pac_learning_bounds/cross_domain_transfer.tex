% Cross Domain Transfer Figure
\begin{figure}[htbp]
\centering
\begin{tikzpicture}[scale=0.85]
    % Define the styles
    \tikzset{
        domain/.style={
            draw,
            fill=blue!15,
            circle,
            minimum size=2cm,
            text width=1.5cm,
            align=center
        },
        concept/.style={
            draw,
            fill=green!15,
            rectangle,
            rounded corners,
            minimum width=1.8cm,
            minimum height=0.8cm,
            text width=1.7cm,
            align=center
        },
        arrow/.style={
            ->,
            thick,
            >=latex
        },
        transfer/.style={
            ->,
            thick,
            dashed,
            >=latex,
            red
        },
        improvement/.style={
            ->,
            thick,
            >=latex,
            blue
        }
    }
    
    % Draw domains
    \node[domain] (d1) at (-3,2) {Domain $d_1$};
    \node[domain] (d2) at (3,2) {Domain $d_2$};
    \node[domain] (d3) at (2,7) {Domain $d_3$};
    \node[domain] (d4) at (7,7) {Domain $d_4$};
    
    % Draw concepts in domain 1
    \node[concept] (c11) at (-4,0) {Concept $c_{1,1}$};
    \node[concept] (c12) at (-2,0) {Concept $c_{1,2}$};
    
    % Draw concepts in domain 2
    \node[concept] (c21) at (2,0) {Concept $c_{2,1}$};
    \node[concept] (c22) at (4,0) {Concept $c_{2,2}$};
    
    % Draw concepts in domain 3
    \node[concept] (c31) at (1,5) {Concept $c_{3,1}$};
    \node[concept] (c32) at (3,5) {Concept $c_{3,2}$};
    
    % Draw concepts in domain 4
    \node[concept] (c41) at (6,5) {Concept $c_{4,1}$};
    \node[concept] (c42) at (8,5) {Concept $c_{4,2}$};
    
    % Connect concepts to domains
    \draw[arrow] (c11) -- (d1);
    \draw[arrow] (c12) -- (d1);
    \draw[arrow] (c21) -- (d2);
    \draw[arrow] (c22) -- (d2);
    \draw[arrow] (c31) -- (d3);
    \draw[arrow] (c32) -- (d3);
    \draw[arrow] (c41) -- (d4);
    \draw[arrow] (c42) -- (d4);
    
    % Cross-domain transfer arrows
    \draw[transfer] (c11) to[bend right] node[midway, below] {Transfer} (c21);
    \draw[transfer] (c12) to[bend left] node[midway, below] {Transfer} (c22);
    \draw[transfer] (c31) to[bend right] node[midway, below] {Transfer} (c41);
    \draw[transfer] (c32) to[bend left] node[midway, below] {Transfer} (c42);
    
    % Meta-knowledge improvement
    \draw[improvement] (d1) to[bend left] node[midway, left] {Meta} (d3);
    \draw[improvement] (d2) to[bend right] node[midway, right] {Meta} (d4);
    
    % Elder universal principle
    \node[draw, fill=purple!15, ellipse, align=center] (elder) at (2,9) {Universal\\Principles};
    \draw[improvement, purple] (elder) -- (d3);
    \draw[improvement, purple] (elder) -- (d4);
    
    % Mentor meta-knowledge
    \node[draw, fill=orange!15, ellipse, align=center] (mentor1) at (-5,4) {Meta-knowledge\\$m_1$};
    \node[draw, fill=orange!15, ellipse, align=center] (mentor2) at (5,4) {Meta-knowledge\\$m_2$};
    
    \draw[improvement, orange] (mentor1) -- (d1);
    \draw[improvement, orange] (mentor1) -- (d3);
    \draw[improvement, orange] (mentor2) -- (d2);
    \draw[improvement, orange] (mentor2) -- (d4);
    
    % Sample complexity labels
    \node[align=left, font=\small] at (-6,0) {Sample complexity:\\$O\left(\frac{VC(C_1)}{\epsilon}\right)$};
    \node[align=left, font=\small] at (6,0) {Sample complexity:\\$O\left(\frac{VC(C_2)}{\epsilon}\right)$};
    
    % Transfer improved sample complexity
    \node[align=left, font=\small, red] at (-6,-1.5) {With transfer:\\$O\left(\frac{VC(C_1) \cdot \alpha}{\epsilon}\right)$\\$\alpha < 1$};
    \node[align=left, font=\small, red] at (6,-1.5) {With transfer:\\$O\left(\frac{VC(C_2) \cdot \alpha}{\epsilon}\right)$\\$\alpha < 1$};
    
    % Title
    \node[align=center, font=\bfseries, scale=1.2] at (2,-3) {Cross-Domain Transfer Improves Sample Complexity};
\end{tikzpicture}
\caption{Cross-domain transfer in Elder framework. Knowledge transfer between similar domains (horizontal arrows) reduces sample complexity by a factor $\alpha$. Meta-knowledge (diagonal arrows) facilitates transfer by identifying mappings between domains. Universal principles (top) further improve transfer by identifying invariant structures across all domains. The Elder system's hierarchical approach combines these mechanisms to achieve better theoretical guarantees for transfer learning compared to traditional approaches.}
\label{fig:cross_domain_transfer}
\end{figure}