% Code example for "Elder, the Arcane Realization"

% Example of Go code for implementing Elder algebra operations
\begin{lstlisting}[language=golang, caption={Implementation of Elder Algebra in GoLang}]
package elder

import (
    "math"
    "errors"
)

// ArcaneElement represents an element in the Elder algebra
type ArcaneElement struct {
    Coefficients []float64  // Coefficients in the standard basis
}

// NewArcaneElement creates a new Arcane element with the given coefficients
func NewArcaneElement(coeffs []float64) *ArcaneElement {
    return &ArcaneElement{
        Coefficients: coeffs,
    }
}

// Add performs addition of two Arcane elements
func (a *ArcaneElement) Add(b *ArcaneElement) (*ArcaneElement, error) {
    if len(a.Coefficients) != len(b.Coefficients) {
        return nil, errors.New("dimensions must match for addition")
    }
    
    result := make([]float64, len(a.Coefficients))
    for i := range a.Coefficients {
        result[i] = a.Coefficients[i] + b.Coefficients[i]
    }
    
    return NewArcaneElement(result), nil
}

// ScalarMultiply multiplies an Arcane element by a scalar
func (a *ArcaneElement) ScalarMultiply(scalar float64) *ArcaneElement {
    result := make([]float64, len(a.Coefficients))
    for i := range a.Coefficients {
        result[i] = scalar * a.Coefficients[i]
    }
    
    return NewArcaneElement(result)
}

// ElderProduct computes the Elder product of two Arcane elements
func (a *ArcaneElement) ElderProduct(b *ArcaneElement) *ArcaneElement {
    n := len(a.Coefficients)
    m := len(b.Coefficients)
    
    // The result will have dimension n+m-1
    result := make([]float64, n+m-1)
    
    for i := 0; i < n; i++ {
        for j := 0; j < m; j++ {
            // Apply the Elder product formula
            sum := 0.0
            for k := 0; k <= min(i, j); k++ {
                binomI := Binomial(i, k)
                binomJ := Binomial(j, k)
                factorial := Factorial(k)
                sum += binomI * binomJ * float64(factorial)
            }
            
            result[i+j-min(i,j)] += a.Coefficients[i] * b.Coefficients[j] * sum
        }
    }
    
    return NewArcaneElement(result)
}

// Norm calculates the Elder norm of an Arcane element
func (a *ArcaneElement) Norm() float64 {
    sum := 0.0
    for i := range a.Coefficients {
        for j := range a.Coefficients {
            gramValue := ElderGramMatrix(i, j)
            sum += a.Coefficients[i] * a.Coefficients[j] * gramValue
        }
    }
    return math.Sqrt(sum)
}

// ElderGramMatrix computes the (i,j) entry of the Elder Gram matrix
func ElderGramMatrix(i, j int) float64 {
    // Implementation of the Elder inner product formula
    if i == j {
        return float64(i + 1)
    } else {
        return float64(min(i, j))
    }
}

// Helper functions
func min(a, b int) int {
    if a < b {
        return a
    }
    return b
}

func Binomial(n, k int) float64 {
    return float64(Factorial(n)) / 
           (float64(Factorial(k)) * float64(Factorial(n-k)))
}

func Factorial(n int) int {
    if n <= 1 {
        return 1
    }
    return n * Factorial(n-1)
}
\end{lstlisting}

% Example of Go code for implementing an Elder space visualization
\begin{lstlisting}[language=golang, caption={Elder Space Visualization Utility}]
package visualization

import (
    "github.com/example/elder"
    "math"
    "image/color"
)

// Point2D represents a point in 2D space
type Point2D struct {
    X, Y float64
}

// MapElderToEuclidean maps a 2D Elder space element to Euclidean coordinates
func MapElderToEuclidean(element *elder.ArcaneElement) (Point2D, error) {
    if len(element.Coefficients) != 2 {
        return Point2D{}, errors.New("element must be in 2D Elder space")
    }
    
    // Apply the mapping from Elder coordinates to Euclidean coordinates
    // This is a simplified mapping for visualization purposes
    x := element.Coefficients[0] * math.Cos(element.Coefficients[1])
    y := element.Coefficients[0] * math.Sin(element.Coefficients[1])
    
    return Point2D{X: x, Y: y}, nil
}

// GenerateElderTrajectory generates a trajectory in Elder space
func GenerateElderTrajectory(start *elder.ArcaneElement, steps int) []Point2D {
    trajectory := make([]Point2D, steps)
    
    current := start
    for i := 0; i < steps; i++ {
        // Apply the Elder evolution operator
        current = ElderEvolution(current)
        
        // Map to Euclidean coordinates for visualization
        point, _ := MapElderToEuclidean(current)
        trajectory[i] = point
    }
    
    return trajectory
}

// ElderEvolution applies one step of the Elder evolution operator
func ElderEvolution(element *elder.ArcaneElement) *elder.ArcaneElement {
    // A simple evolution rule for demonstration purposes
    // In a real application, this would implement the actual Elder dynamics
    a := element.Coefficients[0]
    b := element.Coefficients[1]
    
    newA := a + 0.1*b - 0.01*a*a
    newB := b - 0.1*a - 0.01*b*b
    
    return elder.NewArcaneElement([]float64{newA, newB})
}

// ColorMapElder maps Elder space coordinate to a color
func ColorMapElder(element *elder.ArcaneElement) color.RGBA {
    norm := element.Norm()
    
    // Map the norm to a color (blue to red gradient)
    normNormalized := math.Min(norm/10.0, 1.0)
    r := uint8(255 * normNormalized)
    b := uint8(255 * (1.0 - normNormalized))
    g := uint8(128 * math.Sin(math.Pi*normNormalized))
    
    return color.RGBA{R: r, G: g, B: b, A: 255}
}
\end{lstlisting}
