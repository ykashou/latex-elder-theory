% Graph example for "Elder Theory"

% Example of a basic 2D plot
\begin{figure}[htb]
\centering
\begin{tikzpicture}
\begin{axis}[
    xlabel={$x$},
    ylabel={$\elderstructure{x}$},
    width=10cm,
    height=7cm,
    grid=major,
    xmin=0, xmax=5,
    ymin=0, ymax=8,
    legend pos=north west
]

\addplot[color=DarkSkyBlue, thick, domain=0:5, samples=100]{x^2};
\addlegendentry{$\arcane{1}(x) = x^2$}

\addplot[color=DarkGray, thick, domain=0:5, samples=100]{exp(x/2)};
\addlegendentry{$\arcane{2}(x) = e^{x/2}$}

\addplot[color=LemmaGreen!80, thick, domain=0:5, samples=100]{x^2*sin(deg(x))};
\addlegendentry{$\arcane{3}(x) = x^2\sin(x)$}

\end{axis}
\end{tikzpicture}
\caption{Behavior of the first three Arcane functions}
\end{figure}

% Example of a commutative diagram
\begin{figure}[htb]
\centering
\begin{tikzpicture}[node distance=2.5cm, auto]
\node (X) {$X$};
\node (EX) [right of=X] {$\elder{X}$};
\node (Y) [below of=X] {$Y$};
\node (EY) [right of=Y] {$\elder{Y}$};

\draw[->] (X) to node {$\realization{}$} (EX);
\draw[->] (X) to node [left] {$f$} (Y);
\draw[->] (EX) to node [right] {$\elder{f}$} (EY);
\draw[->] (Y) to node {$\realization{}$} (EY);
\end{tikzpicture}
\caption{Commutative diagram showing the functorial property of Realization}
\end{figure}

% Example of a 3D surface plot
\begin{figure}[htb]
\centering
\begin{tikzpicture}
\begin{axis}[
    width=10cm,
    height=8cm,
    xlabel={$x$},
    ylabel={$y$},
    zlabel={$\realization{z}$},
    view={30}{30},
    colormap={arcane}{color(0)=(DarkSkyBlue) color(1)=(LemmaGreen)}
]

\addplot3[
    surf,
    domain=-2:2,
    domain y=-2:2,
    samples=30,
    samples y=30,
] {exp(-x^2-y^2) * cos(deg(2*x)) * sin(deg(2*y))};

\end{axis}
\end{tikzpicture}
\caption{Surface representation of a 2D Elder Realization}
\end{figure}

% Example of a phase portrait
\begin{figure}[htb]
\centering
\begin{tikzpicture}
\begin{axis}[
    width=10cm,
    height=8cm,
    xlabel={$\arcane{1}$},
    ylabel={$\arcane{2}$},
    grid=major,
    xmin=-2, xmax=2,
    ymin=-2, ymax=2
]

% Vector field representing Elder flow
\addplot[
    ->,
    DarkGray,
    thick,
    quiver={
        u={x*y-y},
        v={-x-y^2},
        scale arrows=0.2,
    },
    samples=15,
    domain=-2:2,
    domain y=-2:2
] {0};

% Critical points
\addplot[only marks, mark=*, mark size=3pt, DarkSkyBlue] coordinates {
    (0,0)
    (1,1)
    (-1,-1)
};

% Sample trajectory
\addplot[TheoremBlue, thick, domain=0:6.28, samples=100, variable=\t]
    ({cos(\t r) * exp(-\t/3)}, {sin(\t r) * exp(-\t/3)});

\end{axis}
\end{tikzpicture}
\caption{Phase portrait of an Elder dynamical system showing critical points and a trajectory}
\end{figure}

% Example of a bifurcation diagram
\begin{figure}[htb]
\centering
\begin{tikzpicture}
\begin{axis}[
    width=12cm,
    height=8cm,
    xlabel={Parameter $\mu$},
    ylabel={Equilibrium $\arcane{*}$},
    grid=major,
    xmin=0, xmax=4,
    ymin=-2, ymax=2
]

% Stable branch
\addplot[DarkSkyBlue, thick, domain=0:4, samples=100]{sqrt(x-1) * (x>=1)};
\addplot[DarkSkyBlue, thick, domain=0:4, samples=100]{-sqrt(x-1) * (x>=1)};

% Unstable branch
\addplot[DarkGray, thick, dashed, domain=0:1, samples=100]{0};
\addplot[DarkGray, thick, dashed, domain=1:4, samples=100]{0};

% Critical point
\addplot[only marks, mark=*, mark size=3pt, LemmaGreen!80] coordinates {
    (1,0)
};

\node[anchor=north west] at (axis cs:1.1,0.1) {Bifurcation point};
\node[anchor=north west] at (axis cs:3,1) {Stable branch};
\node[anchor=north west] at (axis cs:2,-0.1) {Unstable branch};

\end{axis}
\end{tikzpicture}
\caption{Bifurcation diagram for the Elder equation $\dot{\arcane{}} = \mu \arcane{} - \arcane{}^3$}
\end{figure}

% Example of a complex network graph
\begin{figure}[htb]
\centering
\begin{tikzpicture}[
    scale=1.2,
    arcane node/.style={circle, draw=DarkSkyBlue, fill=DarkSkyBlue!20, thick, minimum size=0.8cm},
    elder node/.style={rectangle, draw=LemmaGreen!80, fill=LemmaGreen!20, thick, minimum size=0.7cm},
    link/.style={thick},
    strong link/.style={thick, DarkSkyBlue}
]

% Nodes
\node[arcane node] (A1) at (0,0) {$A_1$};
\node[arcane node] (A2) at (2,1) {$A_2$};
\node[arcane node] (A3) at (4,0) {$A_3$};
\node[arcane node] (A4) at (6,1) {$A_4$};
\node[arcane node] (A5) at (8,0) {$A_5$};

\node[elder node] (E1) at (1,-1.5) {$E_1$};
\node[elder node] (E2) at (3,-1.5) {$E_2$};
\node[elder node] (E3) at (5,-1.5) {$E_3$};
\node[elder node] (E4) at (7,-1.5) {$E_4$};

% Links
\draw[link] (A1) -- (A2);
\draw[link] (A2) -- (A3);
\draw[link] (A3) -- (A4);
\draw[link] (A4) -- (A5);

\draw[strong link] (A1) -- (E1);
\draw[strong link] (A2) -- (E1);
\draw[strong link] (A2) -- (E2);
\draw[strong link] (A3) -- (E2);
\draw[strong link] (A3) -- (E3);
\draw[strong link] (A4) -- (E3);
\draw[strong link] (A4) -- (E4);
\draw[strong link] (A5) -- (E4);

\draw[link, dashed] (E1) -- (E2);
\draw[link, dashed] (E2) -- (E3);
\draw[link, dashed] (E3) -- (E4);

\end{tikzpicture}
\caption{Network representation of Arcane-Elder interactions}
\end{figure}
