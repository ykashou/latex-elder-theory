% Theorem style definitions for "Elder, the Arcane Realization"

% Define theorem styles with tcolorbox
\tcbuselibrary{theorems, skins, breakable}

% Shared options for all theorem environments
\tcbset{
    enhanced,
    breakable,
    colback=white,
    fonttitle=\bfseries,
    fontupper=\normalfont,
    sharp corners,
    boxrule=0.4pt,
    attach boxed title to top left={xshift=0.5cm, yshift=-0.5mm},
    boxed title style={
        sharp corners,
        size=small,
        colback=white,
        frame hidden
    },
    separator sign={:},
    description delimiters={[}{]},
    coltitle=black,
    before skip=10pt,
    after skip=10pt,
}

% Define theorem style
\newtcbtheorem[number within=chapter]{theorem}{Theorem}{
    colback=TheoremBlue!20,
    colframe=TheoremBlue!80,
    after={\hfill\ensuremath{\square}}
}{thm}

% Define lemma style
\newtcbtheorem[number within=chapter]{lemma}{Lemma}{
    colback=LemmaGreen!20,
    colframe=LemmaGreen!80,
    after={\hfill\ensuremath{\square}}
}{lem}

% Define proposition style
\newtcbtheorem[number within=chapter]{proposition}{Proposition}{
    colback=PropositionYellow!20,
    colframe=PropositionYellow!80,
    after={\hfill\ensuremath{\square}}
}{prop}

% Define corollary style
\newtcbtheorem[number within=chapter]{corollary}{Corollary}{
    colback=TheoremBlue!10,
    colframe=TheoremBlue!70,
    after={\hfill\ensuremath{\square}}
}{cor}

% Define definition style
\newtcbtheorem[number within=chapter]{definition}{Definition}{
    colback=DefinitionPurple!20,
    colframe=DefinitionPurple!80,
}{def}

% Define example style
\newtcbtheorem[number within=chapter]{examplebox}{Example}{
    colback=ExampleGray!20,
    colframe=ExampleGray!80,
}{exa}

% Define remark style
\newtcbtheorem[number within=chapter]{remark}{Remark}{
    colback=white,
    colframe=DarkGray!50,
    fonttitle=\bfseries\itshape,
}{rem}

% Define proof environment (non-boxed)
\renewenvironment{proof}{%
    \par\vspace{1ex}\noindent{\itshape Proof. }%
}{%
    \hfill\ensuremath{\square}\par\vspace{1ex}%
}

% Define notation style
\newtcbtheorem[number within=chapter]{notation}{Notation}{
    colback=white,
    colframe=DarkGray!30,
    fonttitle=\bfseries,
}{not}

% Define conjecture style
\newtcbtheorem[number within=chapter]{conjecture}{Conjecture}{
    colback=PropositionYellow!10,
    colframe=PropositionYellow!70,
}{conj}

% Define axiom style
\newtcbtheorem[number within=chapter]{axiom}{Axiom}{
    colback=DefinitionPurple!30,
    colframe=DefinitionPurple!90,
    fonttitle=\bfseries,
}{ax}

% Define claim style
\newtcbtheorem[number within=chapter]{claim}{Claim}{
    colback=LemmaGreen!10,
    colframe=LemmaGreen!70,
}{clm}

% Define observation style
\newtcbtheorem[number within=chapter]{observation}{Observation}{
    colback=ExampleGray!10,
    colframe=ExampleGray!70,
    fonttitle=\bfseries\itshape,
}{obs}

% Define note style (for side notes)
\newtcbtheorem[number within=chapter]{note}{Note}{
    colback=white,
    colframe=DarkSkyBlue!50,
    fonttitle=\bfseries\itshape,
    boxrule=0.4pt,
    left=2pt,
    right=2pt,
    top=2pt,
    bottom=2pt,
    boxsep=2pt,
}{note}
