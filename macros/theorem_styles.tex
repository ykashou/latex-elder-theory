% Theorem and Environment Styling for Elder Framework
% This file defines the visual appearance of theorems, definitions, examples, etc.

% Load required packages
\usepackage{amsthm}
\usepackage{thmtools}
\usepackage{tcolorbox}
\tcbuselibrary{theorems}

% Define color scheme
\definecolor{TheoremBlue}{RGB}{230, 236, 245}
\definecolor{LemmaGreen}{RGB}{230, 245, 230}
\definecolor{PropositionYellow}{RGB}{245, 245, 230}
\definecolor{DefinitionPurple}{RGB}{240, 230, 245}
\definecolor{ExampleGray}{RGB}{240, 240, 240}
\definecolor{NotationColor}{RGB}{255, 240, 245}

% Create theorem styles with tcolorbox
\tcbset{
    theoremstyle/.style={
        enhanced,
        breakable,
        colback=#1!10!white,
        colframe=#1!85!black,
        fonttitle=\bfseries,
        coltitle=black,
        colbacktitle=#1!20!white,
        attach title to upper=\par\smallskip,
        top=2mm,
        bottom=2mm,
        left=4mm,
        right=4mm,
        arc=1mm,
        before skip=8pt,
        after skip=8pt
    }
}

% Define theorem-like environments
\declaretheorem[style=tcbtheorem, 
                tcb={theoremstyle=TheoremBlue}, 
                name=Theorem,
                numberwithin=chapter]{theorem}

\declaretheorem[style=tcbtheorem, 
                tcb={theoremstyle=LemmaGreen}, 
                name=Lemma,
                sibling=theorem]{lemma}

\declaretheorem[style=tcbtheorem, 
                tcb={theoremstyle=PropositionYellow}, 
                name=Proposition,
                sibling=theorem]{proposition}

\declaretheorem[style=tcbtheorem, 
                tcb={theoremstyle=TheoremBlue}, 
                name=Corollary,
                sibling=theorem]{corollary}

\declaretheorem[style=tcbtheorem, 
                tcb={theoremstyle=DefinitionPurple}, 
                name=Definition,
                numberwithin=chapter]{definition}

\declaretheorem[style=tcbtheorem, 
                tcb={theoremstyle=ExampleGray}, 
                name=Example,
                numberwithin=chapter]{example}

\declaretheorem[style=tcbtheorem, 
                tcb={theoremstyle=ExampleGray}, 
                name=Remark,
                numberwithin=chapter]{remark}

% Non-numbered environments
\declaretheorem[style=tcbtheorem, 
                tcb={theoremstyle=NotationColor}, 
                name=Notation,
                numbered=no]{notation}

\declaretheorem[style=tcbtheorem, 
                tcb={theoremstyle=ExampleGray}, 
                name=Algorithm,
                numberwithin=chapter]{algorithm}

% Optional: proof environment styling
\renewenvironment{proof}{
  \pushQED{\qed}
  \normalfont \topsep6\p@\@plus6\p@\relax
  \trivlist\item[\hskip\labelsep
  \itshape\sffamily{Proof.}]\mbox{}\newline
}{
  \popQED\endtrivlist\@endpefalse
}

% Custom "Elder Theorem" environment for especially important results
\declaretheorem[style=tcbtheorem, 
                tcb={theoremstyle=TheoremBlue, 
                     colframe=DarkSkyBlue!90!black,
                     leftrule=3mm}, 
                name={Elder Theorem},
                numberwithin=chapter]{eldertheorem}

% List of theorems configuration
\declaretheoremstyle[
  spaceabove=6pt, 
  spacebelow=6pt,
  notefont=\bfseries, 
  notebraces={[}{]},
  bodyfont=\itshape,
]{thmstyle}

% Table of theorems setup
\newcommand{\listofeldertheorems}{
  \chapter*{List of Principal Results}
  \markboth{LIST OF PRINCIPAL RESULTS}{LIST OF PRINCIPAL RESULTS}
  \addcontentsline{toc}{chapter}{List of Principal Results}
  \begin{description}
    \item[Theorems] provide the main mathematical results in the Elder framework.
    \item[Lemmas] are supporting propositions used to build toward major theorems.
    \item[Propositions] state important facts with complete proofs.
    \item[Definitions] establish the precise meaning of mathematical concepts.
    \item[Examples] demonstrate the application of theoretical concepts.
  \end{description}
  \listoftheorems[ignoreall,show={theorem,eldertheorem}]
}

% Theorem reference formatting
\renewcommand{\thetheoremrefs}{\thechapter.\arabic{theorem}}
\renewcommand{\thedefinitionrefs}{\thechapter.\arabic{definition}}